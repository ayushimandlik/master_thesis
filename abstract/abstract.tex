\documentclass[../main/thesis_msc.tex]{subfiles}

\begin{document}
    \chapter*{Abstract}
    \addcontentsline{toc}{chapter}{Abstract}
    \phantomsection
    \thispagestyle{plain}


This thesis presents the first arcsec resolution observations of two nearby spiral galaxies- IC\,342 and NGC\,628, using LOw Frequency ARray (LOFAR). Low frequency (145 MHz) continuum observations trace older, lower-energy cosmic-ray electrons in the galaxy via radio synchrotron emission. These electrons are ones that have travelled away from their locations of origin (supernova remnants) and are present in the extended disks and haloes of the galaxies. Low frequency surveys also help us understand the properties of warm ionized mediums of galaxies with the help of free-free absorption of synchrotron flux density by the ionized gas. The first part of my thesis presents FACTOR, which is the novel algorithm used to calibrate and image the galaxies using LOFAR visibilities. Images of the two face-on star forming spiral galaxies are presented at various resolutions. This is followed by an analysis of the spectral index distribution in the galaxies obtained between the LOFAR maps and VLA\footnote{Very Large Array} maps of 1.49 GHz for IC\,342 and 3 GHz for NGC 628. Correlations between spectral index values and the levels of star formation are obtained with the help of far-infrared, H-alpha and thermal radio emission maps, which trace the regions of star formation in galaxies. Thermal fractions required to explain the observed spectral indices are calculated. The values of thermal fraction needed to explain the flat spectral index values seen in the central part and arms of IC\,342 seem unphysical. This leads to the explanation that IC\,342 suffers from higher levels of thermal absorption due to higher star formation rates. On the other hand, NGC 628 does not seem to suffer much from thermal absorption owing to its lower levels of star formation. Low frequency observations are also ideal to study diffuse magnetic fields in galaxies. The technique of RM Synthesis is applied to obtain the parallel component of magnetic fields (while its perpendicular counterpart can be calculated from the synchrotron flux density). As a result of wavelength dependent Faraday depolarization, diffuse polarized emission cannot be detected for either of the two galaxies.

    \clearpage
    \thispagestyle{empty}
\end{document}
